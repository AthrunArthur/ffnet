%% LyX 2.0.2 created this file.  For more info, see http://www.lyx.org/.
%% Do not edit unless you really know what you are doing.
\documentclass{article}
\usepackage[latin9]{inputenc}
\usepackage{calc}

\makeatletter
%%%%%%%%%%%%%%%%%%%%%%%%%%%%%% User specified LaTeX commands.


\newcommand{\ffnet}{\textsf{ffnet}}
\newcommand{\ffnetdir}[1]{\textit{/ffnet/root/dir${#1}$}}
\author{Athrun Arthur}
\title{A Quick Start for \ffnet{}}

\makeatother

\begin{document}
\maketitle{}\tableofcontents{}\newpage{}


\section{Introduction}


\subsection{Why \ffnet}

Network programming is a very complicated thing. Of course it's simple
to write a simple ping-pong network application using socket. But
you have to consider many other possible situations in productive
applications. 

Consider ping-pong as an example. There is a server which replys pong
message when receives ping message, and a client which replys ping
message when receives pong message. To make the infinite loop start,
the client need to send the first ping message when the connection
is established. Now let's see what you need to consider if ping-pong
is a product which means avalibility, scalibility, strong and easy
to extend.
\begin{itemize}
\item Connection management. It's obvious as there may be multiple clients.
When a client is offline, the server need to know that.
\item Avalibility. Network is complex partly because you may receive any
possible messages, legal or illegal. You must distinguish those illegal
messages from raw messages. This means you may need to handle some
hostile attacks, like DDoS.
\item Performance. It's a good practice to consider response time in network
programming although ping-pong is simple. Maybe you know proactor
pattern, reactor pattern, asynchronized I/O (like boost.asio), parallel
programming. But you may turn a blind to these solutions because of
complexity 
\item Maintainable. Code refactoring is a normal thing in network programming.
There are many situations recall code refactoring, for example, new
business requirements, unexpected network behaviors, performance tunes
and security ensurance. Again, you turn a blind to possible design
patterns because of complexity.
\item Configurable. Another burdern to adjust very network enviornments.
\end{itemize}
There are many network libraries which aim to bring simple and powerful
network programming, like boost.asio, protocol buffer from Google,
ACE and mudo. But \ffnet{} is aim to provide higher level network
programming enviornment with parallel, asynchronization, security
, debugging and configurable features.


\subsection{\ffnet{} is ...}

\ffnet{} is a opensource framework for network programming in C++.
It's based on boost.asio and provide network management, package serialization
and deserialization, asynchronization, security, debugging and configurable
features. Now \ffnet{} is still under heavy development.


\subsection{Build \ffnet{}}

\ffnet{} uses CMake to organize its source code and it depends on
Boost (1.40 or higher). Suppose you have got the source code of \ffnet{}
and the directory is \ffnetdir{/}. Here are the steps you need to
build it.
\begin{enumerate}
\item cd \ffnetdir{/build}
\item cmake ../
\item make
\end{enumerate}
If you didn't install Boost into system path, you will get an error
in the second step. In this case, you need to specify path of you
Boost in \ffnetdir{/CMakeLists.txt}, like this.

\framebox{\begin{minipage}[t]{1\columnwidth}%
set(CMAKE\_INCLUDE\_PATH \$\{CMAKE\_INCLUDE\_PATH\} ``/home/athrun/boost\_1\_46\_1'')

set(CMAKE\_LIBRARY\_PATH \$\{CMAKE\_LIBRARY\_PATH\} ``/home/athrun/boost\_1\_46\_1/stage/lib'') %
\end{minipage}}You can find two generated file in \ffnetdir{/lib} now if you have
built \ffnet{} successfully. One is a static library and the other
is a shared library.


\subsection{An example}


\section{Use \ffnet}


\section{Inside \ffnet}


\section{Furthur Help}
\end{document}
