%% LyX 2.0.3 created this file.  For more info, see http://www.lyx.org/.
%% Do not edit unless you really know what you are doing.
\documentclass{article}
\usepackage[latin9]{inputenc}
\usepackage{calc}
\usepackage{hyperref}

\makeatletter
%%%%%%%%%%%%%%%%%%%%%%%%%%%%%% User specified LaTeX commands.


\newcommand{\ffnet}{\textsf{ffnet}}
\newcommand{\ffnetdir}[1]{\textit{/ffnet/root/dir${#1}$}}
\newcommand{\pingpong}{\textsf{ping\_pong}}
\newcommand{\pingpongdir}[1]{\textit{/ffnet/root/dir/example/ping\_pong${#1}$}}
\newcommand{\ffnetcomp}[1]{\texttt{#1}}
\author{Athrun Arthur}
\title{A Quick Start for \ffnet{}}

\makeatother

\begin{document}
\maketitle{}\tableofcontents{}\newpage{}


\section{Introduction}


\subsection{Why \ffnet}

Network programming is a very complicated thing. Of course it's simple
to write a simple ping-pong network application using socket. But
you have to consider many other possible situations in productive
applications. 

Consider ping-pong as an example. There is a server which replys pong
message when receives ping message, and a client which replys ping
message when receives pong message. To make the infinite loop start,
the client need to send the first ping message when the connection
is established. Now let's see what you need to consider if ping-pong
is a product which means avalibility, scalibility, strong and easy
to extend.
\begin{itemize}
\item Connection management. It's obvious as there may be multiple clients.
When a client is offline, the server need to know that.
\item Avalibility. Network is complex partly because you may receive any
possible messages, legal or illegal. You must distinguish those illegal
messages from raw messages. This means you may need to handle some
hostile attacks, like DDoS.
\item Performance. It's a good practice to consider response time in network
programming although ping-pong is simple. Maybe you know proactor
pattern, reactor pattern, asynchronized I/O (like boost.asio), parallel
programming. But you may turn a blind to these solutions because of
complexity 
\item Maintainable. Code refactoring is a normal thing in network programming.
There are many situations recall code refactoring, for example, new
business requirements, unexpected network behaviors, performance tunes
and security ensurance. Again, you turn a blind to possible design
patterns because of complexity.
\item Configurable. Another burdern to adjust very network enviornments.
\end{itemize}
There are many network libraries which aim to bring simple and powerful
network programming, like boost.asio, protocol buffer from Google,
ACE and mudo. But \ffnet{} is aim to provide higher level network
programming enviornment with parallel, asynchronization, security
, debugging and configurable features.


\subsection{\ffnet{} is ...}

\ffnet{} is a opensource framework for network programming in C++.
It's based on boost.asio and provide network management, package serialization
and deserialization, asynchronization, security, debugging and configurable
features. Now \ffnet{} is still under heavy development.


\subsection{Build \ffnet{}}

\ffnet{} uses CMake to organize its source code and it depends on
Boost (1.49 or higher). Suppose you have got the source code of \ffnet{}
and the directory is \ffnetdir{/}. Here are the steps you need to
build it.
\begin{enumerate}
\item cd \ffnetdir{/build}
\item cmake ../
\item make
\end{enumerate}
We don't recommend you to use the Boost in your system path, so you need to specify path of your
Boost in \[\ffnetdir{/CMakeLists.txt}\].
Now you can find two generated file in \ffnetdir{/lib} now if you have
built \ffnet{} successfully. One is a static library and the other
is a shared library.

\ffnet{} also has some examples to show how to use it. Building examples is similar as building \ffnet{},
that means you need to specify the necessary Boost path in the examples's CMakeLists.txt.


\section{Use \ffnet}
In this section, we will write an example to illustrate how to use \ffnet{}. The example is about ping-pong between 
a server and a client and it's based on TCP connection. The server is listening until a client launch a TCP connection.
Once the connection is built, the client sends a \textit{ping} message. The server will send a \textit{pong} message as 
an response to each \textit{ping} message. Then the client will get the \textit{pong}
message and re-send another another \texttt{ping} message. The progress will continue and never stop.

The first step to use \ffnet{} is to include necessary header files.
\begin{verbatim}
 #include <network.h>
\end{verbatim}
The next step should be declaring necessary messages, ping and pong. You can declare message 
by inheriting \texttt{ffnet::Package} and implementing \texttt{virtual void archive(ffnet::Archive \& ar);}.
It's not complicated, but boring. So I prefer google's ProtoBuf and you can get more details about it from google.
Now the messages are declared like this:
\begin{verbatim}
package PingPong;

message Ping{
required string msg = 1;
required int32 id = 2;
}

message Pong{
required string msg = 1;
required int32 id = 2;
}
\end{verbatim}

Now it's time to initialize our network enviornment. In \ffnet{}, network is seen as a group of nervures. Each nervure can have
a TCP server, several TCP connections, and several UDP connections. Each nervure is a single thread. So the only reason to use
multiple nervures is for concurrency. To use ProtoBuf in \ffnet{}, we need \texttt{ffnet::ProtoBufNervure}.

\subsection{Server}
In the server side, the network is initialized like this:
\begin{verbatim}
 ffnet::ProtoBufNervure pbn;
 pbn.initTCPServer(10005);
\end{verbatim}

Now we need to receive \textit{ping} message and do something. There is no need to handle buffer and deserialization. 
The message \ffnet{} gives us will be a deserialized \textit{ping}. We just specify a callback handler for it.
\begin{verbatim}
 pbn.addNeedToRecvPkg<PingPong::Ping>(onRecvPing);
\end{verbatim}
Remind that \texttt{PingPong::Ping} is defined by ProtoBuf. \texttt{addNeedToRecvPkg} has the following prototype:
\begin{verbatim}
 template<class MsgTy_>
 void addNeedToRecvPkg(
     boost::function<void (
         boost::shared_ptr<MsgTy_>,  //pMsg
         ffnet::EndpointPtr_t) > //pEP
     );
\end{verbatim}
, in which \texttt{pMsg} is the received message and \texttt{pEP} is the source endpoint. Now we can implement 
\texttt{onRecvPing} and response a \textit{pong} message.
\begin{verbatim}
 void onRecvPing(boost::shared_ptr<PingPong::Ping> pPing,
                 ffnet::EndpointPtr_t pEP)
{
    std::cout<<pPing->msg()<<std::endl;

    boost::shared_ptr<PingPong::Pong> pkg(new PingPong::Pong());
    pkg->set_msg("Pong from server!");
    pkg->set_id(0);
    ffnet::NetNervure::send(pkg, pEP);
}
\end{verbatim}
Finally the left thing is to run the nervure thread.
\begin{verbatim}
 pbn.run();
\end{verbatim}

\subsection{Client}



In the client side, the network is initialized like this by specifying remote endpoint.
\begin{verbatim}
 ffnet::ProtoBufNervure pbn;
 ffnet::EndpointPtr_t pEP(
     new ffnet::Endpoint
         (ffnet::tcp_v4,
          boost::asio::ip::address_v4::from_string("127.0.0.1),
          10005)
     );
 pbn.addTCPClient(ep);
\end{verbatim}
Similar as server side, we need to specify \texttt{PingPong::Pong}'s callback handler.
\begin{verbatim}
 pbn.addNeedToRecvPkg<PingPong::Pong>(onRecvPong);
 
 void    onRecvPong(
 boost::shared_ptr<PingPong::Pong>pPong, 
 ffnet::EndpointPtr_t pEP)
{
    std::cout<<"got pong! "<<pPong->msg();
    sendPingMsg(pEP);
}
\end{verbatim}

Besides, we also need to send message until connection is built. \ffnet{} provides \texttt{event} to do that. There are three
basic events, \texttt{tcp\_get\_connection}, \texttt{tcp\_lost\_connection} and \texttt{udp\_send\_recv\_exception}.
Here we need to response \texttt{tcp\_get\_connection}.
\begin{verbatim}
 using namesapce ffnet::event;
 ...
 Event<tcp_get_connection>::listen(
     &pbn, onConnSucc);
 ...    
 void    onConnSucc(ffnet::TCPConnectionBase *pConn)
{
    std::cout<<"connect success"<<std::endl;
    ffnet::EndpointPtr_t tpp(
        new ffnet::Endpoint(pConn->getSocket().remote_endpoint()));
    sendPingMsg(tpp);
}
\end{verbatim}

\subsection{Complete Code of PingPong}
Please check the example \texttt{pb\_tcp\_pingpong} in \ffnet{}.

\section{Inside \ffnet}
\subsection{Perspective from {\ffnet}}
The authors of {\ffnet} had awful experiences on network programming, including native socket programming, boost.asio, and some other 3rd party network libraries. To let readers know the awful points clearly, we'd like to list some of them randomly instead of sorting them and listing them all here.
\begin{itemize}
  \item Defining network packages is quite error prone. There are several aspects when I say error-prone, 1), inherit from package base class and implement too much methods (like serialize and deserialize); it's really difficult to make sure all code you are writing are consistent and correct; 2), need to decide how to serialize a package, or you loose the flexibility. In native socket programming or boost.asio, you definitely need to define all packages yourself. In 3rd party, it's not always flexible enough to define your own packages.
  \item Exploiting parallelism. In most network programming practice, it's developer's responsibility to do threading. Actually, that's quite stupid as it's well known about how to exploit parallelism and there is no reason to do it again and again.
  \item Event handling is anti-intuitive. Although receiving-handling-sending is quite easy to program, event handling is quite weired in some libraries.
  \item Type converting or type safe. We are crazy C++ fans, and we like type safe as that can make sure our programs are strong. However, we have to do type converting in many libraries, like converting a base class pointer to a concrete class pointer (this happens when handling a received package pointer). 
\end{itemize}
In our experience, even a line of converting type in C++ may cause stupid bugs. After encountering too many stupid bugs, we realize we need another library to make ourself happy. Thus we start rethinking what a network programming library actually do. It will be very interesting to share our thoughts with you readers.
In our perspective, a network programming library should
\begin{itemize}
  \item enable users to handle receiving and send packages,
  \item enable users to handle any events they want with least overhead,
  \item enable users to exploit parallelism easily,
  \item easy to extend on both features and packages,
  \item static type-checking,
  \item really clean and intuitive programming interface,
  \item a logging subsystem with low overhead,
  \item portable on mainstream systems.
\end{itemize}

Finally, we have {\ffnet}, another network programming library in the world. The most exciting thing is that {\ffnet} satisfies all requirements in our mind. We hope you have found that in Section 2, and we're happy that you are still eager to know how {\ffnet} achieve that.

\subsection{Architure of {\ffnet}}

\subsection{\ffnetcomp{NetNervure} --- The Core of {\ffnet}}
\subsection{\ffnetcomp{Package}s --- Extend {\ffnet}}
\subsection{\ffnetcomp{BondAndSplitter} --- How to Serialize and Deserialize}
\subsection{\ffnetcomp{Event} --- Being Active}
\subsection{\ffnetcomp{Log} --- Recording and Debuging}
\section{Known Issues}
In these section, we'd like to record some issues which are mainly caused by 3rd party softwares.
\subsection*{Boost1.55 on Mac OS}
This is a compile error. Typically, the error looks like this
\begin{verbatim}
/usr/local/include/boost/atomic/detail/gcc-atomic.hpp:983:22: error: no viable conversion from 'int' to 'storage_type' (aka 'boost::atomics::detail::storage128_type')
        storage_type tmp = 0;
                     ^     ~
/usr/local/include/boost/atomic/detail/gcc-atomic.hpp:932:28: note: candidate constructor (the implicit copy constructor) not viable: no known conversion from 'int' to 'const boost::atomics::detail::storage128_type &' for 1st
      argument
struct BOOST_ALIGNMENT(16) storage128_type
                           ^
/usr/local/include/boost/atomic/detail/gcc-atomic.hpp:932:28: note: candidate constructor (the implicit move constructor) not viable: no known conversion from 'int' to 'boost::atomics::detail::storage128_type &&' for 1st
      argument
struct BOOST_ALIGNMENT(16) storage128_type
                           ^
\end{verbatim}
Basically, you need to patch the boost header file. Please check \url{https://github.com/Homebrew/homebrew/issues/27396}.

\subsection*{Using Clang on Mac}
If you are using Clang on Mac, you can build everything. But the problem is getting "Segment fault 11" when you execute the binary.

The solution is just using GCC.
\section{Furthur Help}
If you have any problems, please send e-mail to \texttt{athrunarthur@gmail.com}.
\end{document}
